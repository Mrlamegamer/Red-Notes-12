% !TEX root = ./physics.tex

\chapter{Module 5 \; Advanced Mechanics}

\section{Practical Investigation 5.1}
	Aim: To confirm the dependence of the range of a projectile (the horizontal distance it travels) on its time of flight and launch velocity by predicting the landing point of a projectile and then testing the prediction.

	\subsection{Method}
		\begin{enumerate}
			\item Mark a point on the ramp, where the ball can be consistently released from the same position
			\item Measure the horizontal distance from the end of the ramp to the edge of the table
			\item Release the ball and time how long it takes to travel the horizontal distance from the base of the ramp to the edge of the table
			\item Repeat the test three times
		\end{enumerate}

	\subsection{Data and Analysis}
		\begin{enumerate}
			\item Is the velocity being calculated the velocity of the ball at the edge of the table? If not, is it a reasonable approximation? Explain your answer.
			\item What effect would increasing the horizontal distance have on the reliability of your measurements?
			\item Calculate the time it takes for the ball to fall from the table to the floor
			\item Calculate the distance the ball will travel in the horizontal direction, ie. the range.
		\end{enumerate}

	\subsection{Conclusion}
		\begin{enumerate}
			\item State whether your prediction was successful, and describe any difficulties encountered in testing the prediction.
			\item In this experiment, the assumption was made that there is negligible effect from air resistance. Would the effect of air resistance be more significant if the ball was released from a height of 30 cm up the ramp or 15 cm? Explain.
			\item What is the major source of error in this experiment? What steps were taken to minimise it?
		\end{enumerate}


\section{Practical Investigation 5.2 - The effect of launch angle on range}

	Aim: To investigate the relationship between the launch angle of a projectile, its motion and the range of the projectile. \;

90cm vertical height
	\subsection{Method}
		\begin{enumerate}
			\item Set the launcher in the horizontal position with a launch angle of 0\textdegree.
			\item Load a projectile into the launcher and ensure that the launcher is set to its maximum compression or distance setting.
			\item Launch the projectile, and note the point of impact on the paper.
			\item Lay a sheet of carbon paper on top of the white paper over the point of impact, carbon side down, so that when a ball lands on it there will be a mark on the paper.
			\item Place a sheet of paper where the ball hits. Highlight the point with a pencil or marker when the projectile lands.
			\item State recording with the data collection system and launch the projectile
			\item Use the angle indicator on the launcher the angle of inclination by 10\textdegree each time for data points between 10\textdegree and 80\textdegree.
			\item Measure the horizontal velocity for each angle.
		\end{enumerate}

	\subsection{Results}
		\begin{table}[htbp] \label{5.2 Table of Results}
			\centering
			\begin{tabular}{l|l|l}
				Angle (\textdegree) & Time (s)	& Horizontal Range (m)  \\ \hline
				0 					& 0.46		& 1.5 		\\
				0 					& 0.41		& 1.65		\\ \hline
				10					& 0.46		& 1.94		\\
				10					& 0.61		& 2.00		\\
				10					& 0.59		& 1.85		\\ \hline
				20					& 0.61		& 2.03		\\
				20					& 0.65		& 1.98		\\
				20					& 0.58		& 2.003		\\ \hline
				30					& 0.83		& 2.065		\\
				30					& 0.71		& 2.122		\\
				30					& 0.83		& 2.185		\\ \hline
				40					& 0.79		& 2.22		\\
				40					& 0.70		& 2.16		\\
				40					& 0.78		& 2.14		\\ \hline
				50					& 0.85		& 1.99		\\
				50					& 0.83		& 1.845		\\
				50					& 0.93		& 2.031		\\ \hline
				60					& 0.96		& 1.708		\\
				60					& 0.85		& 1.76		\\
				60					& 0.88		& 1.59		\\ \hline
				70					& 0.85		& 1.14		\\
				70					& 0.83		& 1.21		\\
				70					& 0.86		& 1.31		\\ \bottomrule
			\end{tabular}
		\end{table}


\section{Circular Motion} \label{4/11/2024}
\subsection{Uniform Circular Motion}
Occurs when objects travel in a circle at a constant speed, taking the same length of time to make each revolution

The period T is the time taken to travel the full circle

The distance covered during the period depends upon the radius of the circle of travel and is equal to the circle's circumference, ie. $d=2\pi r$



\section{Practical Investigation 5.3 - Circular Motion - centripetal force in a horizontal plane}

	Aim: To investigate the relationship between the centripetal force acting on an object moving in a circle of constant radiusand the frequency of revolution.

\subsection{Method}
	\begin{enumerate}
		\item Securely tie one end of the fishing line to the small, soft mass
		\item Pass the fishing line down through the thin plastic tube and attach a 50g slotted mass carrier to the end.
		\item Attach an alligator clip tot he line to act as a marker for a measured radius of aroudn 1 m.
		\item Spin the stopper in a horizontal circular path at a speed that pulls the paperclip up to, but not touching the bottom of the tube.
		\item Measure time taken for 20 revolutions
		\item Add 50g and repeat steps.
	\end{enumerate}

	\begin{table}[htbp]
		\centering
		\begin{tabular}{llllllllll}
		Mass (kg) & \multicolumn{4}{c}{Time for 20 revolutions (s)} & \multicolumn{1}{l}{Period (s)} & \multicolumn{1}{l}{f=1/T} & \multicolumn{1}{l}{$F_c$ (N)} & \multicolumn{1}{l}{$F_G$ (N)} & \multicolumn{1}{l}{$\omega$ \textdegree$s^{-1}$} \\
				& \multicolumn{1}{l}{Trial 1} & 2     & 3     & \multicolumn{1}{l}{Avg} &       &       &       &       &  \\ \hline
		\multicolumn{1}{r}{0.05} & 10.34 & 10.78 & 12.34 & 11.15 & 0.558 & 1.79  & 0.6601 & 0.49  & 11.27 \\
		\multicolumn{1}{r}{0.1} & 9.00  & 8.97  & 8.85  & 8.94  & 0.447 & 2.24  & 1.027 & 0.98  & 14.06 \\
		\multicolumn{1}{r}{0.15} & 7.93  & 7.22  & 7.25  & 7.47  & 0.373 & 2.68  & 1.473 & 1.47  & 16.83 \\
		\multicolumn{1}{r}{0.2} & 6.56  & 6.47  & 7.03  & 6.69  & 0.334 & 2.99  & 1.837 & 1.96  & 18.79 \\
		\end{tabular}
	\end{table}
	