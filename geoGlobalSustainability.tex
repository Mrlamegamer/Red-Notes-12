% !TEX root = ./geography.tex

\chapter{Global Sustainability \; Aquaculture}

\section{Introduction to Aquaculture} \label{1/11/2024}
	"Farming of aquatic species in controlled or semi-controlled conditions"
		Eg. Salmon, barramundi, lobsters (can be semi-controlled), crabs, prawns, oysters, scallops, seaweed
		Non food: pearl scallops, coral (people keeping pets), crocodile skin
		Pets: goldfish

	In situ $\rightarrow$ In the environment

	Ex situ $\rightarrow$ Isolated to the environment

	Eg. Oyster farms in situ may be affected by external factors like a sewage spill

	\subsection{History}
		Although the Brewarrina fish traps are one of the oldest human constructions, they aren't real farms
		Roman oyster farm
		Chinese carp farm

	Aquaculture is practised across a wide variety of locations and species. Can be:
	\begin{itemize}
		\item Marine (mariculture), estuary or freshwater (in-land)
			\subitem Mariculture is currently underutilised, vast ocean space that isn't being used
		\item In-situ or ex-situ
		\item Fin-fish, crustacean, molluscs, or plants (usually algae)
			\subitem Carp (trash fish)
		\item For human consumption, fishmeal, or fish oil
		\item For local consumption or for export earning
			\subitem Norway and Chile grow the majority of the world's salmon, and is exports
			\subitem Changes the nature that the fish grows
	\end{itemize}

	Aquaculture is \textbf{NOT} fishing

	 In 2018, aquaculture produced 114.5 million tonnes in live weight, with a total farm-gate sale value of US\$263.6 billion
	Aquaculture accounted for 46\% of the total seafood production and 52\% of fish for human consumption
	China produces and consumes the largest amount of aquaculture, but also more broadly Asian countries

	There aren't that many inland waters, so inland fisheries do not have a significant amount of production \footnote{Carp and tilapia are not nice - David Latimer}

	Types of Economic Activity
	\begin{itemize}
		\item Primary - Farming
		\item Secondary - Manufacturing, producing
		\item Tertiary - Distribution of goods, using produced goods
		\item Quanternary - Researcher of salmon
	\end{itemize}

	\subsection{Distribution of Aquaculture}
		Aquaculture is mainly centred around Asia, with China representing aroudn 60\% of global aquaculture
		Fish is common in South-east Asia, especially with river fish eg. Vietnam
		Other countries just catch their fish

		African countries do not have the development or GDP to farm fish. Culturally also doesn't eat fish
		\footnote{"I don't like river fish, it's gross" - David Latimer}

		Developing countries are increasing their share of international fish trade
		Countries with large fishing catches often have larger aquaculture production
		
		Various places have cultural preferences and natural advantages for the production of particular species
		\begin{itemize}
			\item Predominantly carp \footnote{"River fish have a bland, muddy flavour" - David Latimer, D1 river fish hater}
			\item Seaweeds
			\item Tilapia
			\item Oysters
			\item Clams
			\item Catfish
			\item Prawns - Warm species
			\item Salmons, trouts, smelts - Salmon is expensive
			\item Freshwater fishes
		\end{itemize}

		As China gets richer and richer, they will seek to eat more expensive fish, therefore increasing the demand

\section{Draft Nature and Spatial Patterns Text} \label{4/11/2024 - 6/11/2024}
The text below is a reasonable, band 4-5 response to the stimulus prompt \textbf{“Examine the nature and spatial patterns of ONE global economic activity”}. Use the FAO report below to help you edit the text into a strong Band 6 response, complete with a clear thesis, detailed information and vocabulary, and well structured paragraphs.  Your finished text should be around 300-500 words in length. 

{\large Draft Text}

Aquaculture is global economic activity whereby people grow fish for food and trade. Aquaculture takes places around the globe, giving people both food and money. 

Aquaculture is really old, having been practised for years and years. However, people grow lots of different species today. It's important to state that aquaculture and fishing are different activities.

The economic activity of aquaculture can be carried out in both rich and poor countries. However, different countries tend to practise aquaculture differently and for different reasons. Aquaculture is mostly practised in rich countries. 

Aquaculture is also practised in different environments. Moreover, these different types of aquaculture are not growing at the same speed. Some types of aquaculture are growing much more rapidly than others. 

{\large Comments}
\begin{itemize}
	\item Use stats
	\item In an "examine the nature and spatial distribution" question, evenly allocate writing to each part
	\item Specify location; Asia is very broad but aquaculture is focused around only 5
\end{itemize}

\section{Influences on the global economic activity} \label{8/11/2024}
	"How do different things affect the activity of aquaculture"
	Nature, spatial patterns, future changes, sustainability

	Biophysical
	Economic
	Technological
	Political/Organisational

	\begin{table}[htbp]
		\centering
		\begin{tabular}{ll}
			\hline
			Biophysical & How the biophysical environment and ecosystems influence aquaculture \\
			Economic & A \\
			Technological & New tech \\
			Political/Organisational & How is it controlled \\
		\end{tabular}
	\end{table}

	\subsection{Biophysical Factors}
		There are 622 species recognised by the FAO as being produced by aquaculture with each species requiring its own specific biophysical requirements

		Local water conditions can impart "\textbf{merroir}" to seafood $\rightarrow$ the flavour it has

		Local conditions flavour specialisation and give places competitive advantages
		\begin{itemize}
			\item Atlantic Salmon production is dominated by cold deep waters found in Norway and Chile
			\item Salmonids have become the largest single fish commodity by value
			\item Shrimp production benefits from brackish, warm tropical waters
		\end{itemize}

		Ex situ aquaculture attempts to separate aquaculture from the biophysical environment by controlling for temperature and chemistry. However, it is difficult to reproduce the conditions cheaply

		\subsubsection{Water Chemistry}
			The local bedrock and substrates will impact various chemical characteristics to the water, such as nitrates, phosphates, heavy metals
			Heavy metals are present due to mining operations that 

			Salinity is one of the most important characteristics of the water used in aquaculture
			\begin{itemize}
				\item Briny - High salinity
				\item Saline - Seawater, salt lakes
				\item Brackish - Estuaries, mangrove swamps
				\item Fresh - Ponds, lakes, river, streams
			\end{itemize}
		
			Eg. Oyster farmers will move their oysters up and down stream to control the way they grow

			Salmon farms need high flow of water to account for the waste produced by the high concentration of salmon
			Water plants can generally be grown anywhere

			66\% finfish, 22\% crustaceans, 12\% molluscs
			
		\subsubsection{Climate}
			Atlantic Salmon require deep water with temperatures below 10\textdegree C giving Norway and Chile an advantage

			Vannamei Shrimp require brackish, estuarine water that does not fall below 20 \textdegree C giving South-East Asian nations an advantage

		\subsubsection{Ecological}
			Aquaculture can have a highly detrimental interaction with local and global ecologies. For example:
			\begin{itemize}
				\item Carbon emissions from feed catch trawling
				\item By-catch from trawling
				\item Land clearing of mangroves
			\end{itemize}

			Aquaculture ventures often have to work with nearby human settlements. Some communities use this to produce multi-trophic production systems

			Disease outbreaks are increasing in aquaculture due to \textbf{monoculture}

			Eg. An in situ production system:
			\begin{itemize}
				\item Food pellets aren't completely consumed, increasing the concentration of food in an area
				\item The introduction of non-native species that are highly competitive
					\subitem Bad weather can increase the likelihood of escapes
				\item Predators like birds can attack birds, increasing the overall level of fish stress
				\item Bulk antibiotics applied to fish farms can impact resistance in future
					\subitem This can extend to humans consuming the fish 
				\item Fat salmon are better to eat, however to become this way they are overfed and lazy. If salmon escape, they can breed lazy salmon in the natural environment \footnote{"How do you get a fat salmon" - Latimer} \footnote{"You want a fat, lazy salmon" - Latimer}
			\end{itemize}

			To mitigate the greater environmental impacts:
			\begin{itemize}
				\item Make the farm ex situ
				\item Lower the density of the farm (However this lowers profit)
				\item Enironmental Laws $\rightarrow$ Developing nations are also able to use lax environmental laws to develop coastal land for aquaculture
			\end{itemize}

			\textbf{Positive Ecological Impacts}

			Oyster farming industries can filter estuaries and apply pressure to keep waterways clean - encourages community to reduce pollution

		\subsection{Economic Influences}
			\subsubsection{Commodity Prices}
				Variable exchange rates and market prices for export commodities will modify production, including access to feed meal.

				In recent years, other major producing countries have reported low market prices of staple species, reflecting market saturation at least seasonally and locally for these mass-produced species.
				
				Salmon and avocado sushi was invented by Norwegians to encourage Japanese to purchase it
				Before introduction, Japan was not a major salmon consumer but was wealthy and Norwegians had an excess

				The \textbf{commodification} of aquaculture produce also placed demand to exceed environmental capacity. Commodification drives the production of more goods

			\subsubsection{Differences in HIC and LIC aquaculture}
				In some LICs, low labour costs can be a competitive advantage for production. 

				However, capital can be difficult to source in LICs - Greater degree of risk, less willingness for investors

				HICs will use the high value of their markets to demand higher quality produce.

				China has been accused of devaluing its yuan to promote exports. If their exchange rate is lower, their exports are cheaper, people will buy more, better for the Chinese market. (Denied by China ofc)
			
			\subsubsection{Urbanisation}
				People are increasingly living in cities with higher incomes and better infrastructure to facilitate fish purchases

			\subsubsection{Labour Specialisation}
				In HICs, changes in life expectations have made it difficult to find adequate labour - It is difficult to get people to work in far away aquaculture farms. Is promoted in Australia by using Tongan migrants as labour.

				In LICs, small scale farms account for much greater rates of production

				The International Labour Organisation (ILO) has identified that aquaculture utilises child and slave labour, but notes there is limited availability of evidence.

				People from Myanmar run to Thailand to escape the government. However questionable law enforcement in Thailand promotes illegal labour.
				One solution to this is international agreements and tarriffs, however this is unlikely because people want their shrimp cheap. It is then a responsbility of the consumer to check the source of produce
